\documentclass[10pt,conference,compsoc]{IEEEtran}

\usepackage{amsmath}
\usepackage{bm}
\usepackage{datetime}
\usepackage{enumerate}
\usepackage{graphicx}
\PassOptionsToPackage{hyphens}{url}
  \usepackage[colorlinks=true,linkcolor=blue]{hyperref}
\usepackage{import}
\usepackage{listings}
\usepackage{parskip}
\usepackage{tabulary}

% Bibliography setup
\usepackage[style=numeric,
            citestyle=numeric,
            defernumbers=true,
            sorting=nyt,
            sortcites=true,
            autopunct=true,
            autolang=hyphen,
            hyperref=true,
            abbreviate=false,
            backref=true,
            backend=biber]{biblatex}
\addbibresource{sysid-ols.bib}
\defbibheading{bibempty}{}

\newcommand{\includecode}[2][c]{%
  \lstinputlisting[escapechar=, style=custom#1]{#2}
}

% Load after hyperref
\usepackage[style=super,nonumberlist,toc]{glossaries}
\usepackage[notlof, notlot, nottoc]{tocbibind}
\usepackage{tocloft}

% Set up snippet environment
\newcommand{\listsnippetsname}{\Large List of Snippets}
\newlistof{snippets}{los}{\listsnippetsname}
\cftsetindents{snippets}{\parindent}{1.5\parindent}
\newenvironment{snippet}{
  \renewcommand{\caption}[1]{
    \refstepcounter{snippets}
    \begin{center}
      \par\noindent{\footnotesize Snippet \thesnippets. ##1}
    \end{center}
    \addcontentsline{los}{snippets}
      {\protect \numberline{\thesnippets}{\ignorespaces ##1}}
  }
}
{\par\noindent}

\newcommand{\mtx}[1] {\bm #1}

% Disable automatic indent and provide \indent command
\newlength\tindent
\setlength{\tindent}{\parindent}
\setlength{\parindent}{0pt}
\renewcommand{\indent}{\hspace*{\tindent}}

\lstdefinestyle{customMatlab}{
  language=Matlab,
  breaklines=true,
  xleftmargin=0.125in,
  basicstyle=\footnotesize\ttfamily,
  keywordstyle=\color{blue},
  morekeywords=[2]{1}, keywordstyle=[2]{\color{black}},
  identifierstyle=\color{black},
  stringstyle=\color[RGB]{170, 55, 241},
  commentstyle=\color[RGB]{28, 172, 0},
  showstringspaces=false,
  emph=[1]{for,end,break},emphstyle=[1]\color{red},
}

\lstdefinestyle{customcpp}{
  belowcaptionskip=1\baselineskip,
  breaklines=true,
  xleftmargin=0.125in,
  language=C++,
  showstringspaces=false,
  basicstyle=\footnotesize\ttfamily,
  keywordstyle=\color[RGB]{128, 128, 0},
  commentstyle=\color[RGB]{0, 128, 0},
  identifierstyle=\color{black},
  stringstyle=\color[RGB]{0, 128, 0},
}

% Paper header
\usepackage{fancyhdr}
\pagestyle{fancy}
\lhead{System Identification Tool OLS Improvements}
\rhead{\thepage}
\lfoot{}
\rfoot{}
\fancyfoot[C]{}  % remove normal page numbers

\allowdisplaybreaks
\begin{document}

\title{System Identification Tool OLS Improvements}
\author{Tyler Veness}
\maketitle

\section{Introduction}

The feedforward model utilized by the FRC System Identification tool is
\begin{equation}
  u = K_s + K_v v + K_a a \label{eq:u_ff}
\end{equation}

where $u$ is input voltage, $v$ is velocity, $a$ is acceleration, and $K_s$,
$K_v$, and $K_a$ are constants reflecting how much each variable contributes to
the required voltage.

The System Identification tool uses ordinary least squares (OLS) on triplets of
recorded voltage, velocity, and acceleration to obtain $K_s$, $K_v$, and $K_a$.
The problem is we don't know acceleration directly, and the tool's current
method of numerical differentiation of velocity is noisy. Ideally, we could find
the characterization gains with samples of current velocity, current input, next
velocity, and the nominal sample period instead.

\section{Overview}

First, we'll put equation \eqref{eq:u_ff} into the form
$\frac{dx}{dt} = Ax + Bu + c$. Then, we'll discretize it to obtain an equation
of the form $x_{k+1} = \alpha x_k + \beta u_k + \gamma$. Then, we'll perform OLS
on $x_{k+1} = \alpha x_k + \beta u_k + \gamma$ using triplets of current
velocity, next velocity, and input voltage to obtain $\alpha$, $\beta$, and
$\gamma$. Since we solved for those in the discretization step earlier, we have
a system of three equations we can solve for $K_s$, $K_v$, and $K_a$.

\section{Derivation}

First, solve $u = K_s + K_v v + K_a a$ for $a$.
\begin{align*}
  K_a a &= u - K_s - K_v v \\
  a &= \frac{1}{K_a} u - \frac{K_s}{K_a} - \frac{K_v}{K_a} v \\
  a &= -\frac{K_v}{K_a} v + \frac{1}{K_a} u - \frac{K_s}{K_a}
\end{align*}

Let $x = v$, $\frac{dx}{dt} = a$, $A = -\frac{K_v}{K_a}$, $B = \frac{1}{K_a}$,
and $c = -\frac{K_s}{K_a}$.
\begin{equation*}
  \frac{dx}{dt} = Ax + Bu + c
\end{equation*}

Since $Bu + c$ is a constant over the time interval $[0, T)$ where $T$ is the
sample period, this equation can be integrated according to appendix F.2 of
\textit{Controls Engineering in FRC} \cite{bib:controls-in-frc}, which gives
\begin{equation*}
  x_{k+1} = e^{AT} x_k + A^{-1} (e^{AT} - 1) (Bu_k + c)
\end{equation*}

Substitute $A$, $B$, and $c$ back in and simplify.
\begin{equation*}
  \begin{aligned}
    x_{k+1} =& e^{AT} x_k + A^{-1} (e^{AT} - 1)Bu_k + A^{-1} (e^{AT} - 1)c \\
    x_{k+1} =& e^{-\frac{K_v}{K_a} T} x_k \\
      +& \left(-\frac{K_v}{K_a}\right)^{-1}
        \left(e^{-\frac{K_v}{K_a} T} - 1\right)
        \left(\frac{1}{K_a}\right) u_k \\
      +& \left(-\frac{K_v}{K_a}\right)^{-1}
        \left(e^{-\frac{K_v}{K_a} T} - 1\right)
        \left(-\frac{K_s}{K_a}\right) \\
    x_{k+1} =& e^{-\frac{K_v}{K_a} T} x_k \\
      -& \frac{K_a}{K_v}
        \left(e^{-\frac{K_v}{K_a} T} - 1\right)
        \frac{1}{K_a} u_k \\
      +& \frac{K_a}{K_v}
        \left(e^{-\frac{K_v}{K_a} T} - 1\right)
        \frac{K_s}{K_a} \\
    x_{k+1} =& e^{-\frac{K_v}{K_a} T} x_k \\
      +& \frac{1}{K_v}
        \left(1 - e^{-\frac{K_v}{K_a} T}\right) u_k \\
      +& \frac{K_s}{K_v}
        \left(e^{-\frac{K_v}{K_a} T} - 1\right)
  \end{aligned}
\end{equation*}

This equation has the form $x_{k+1} = \alpha x_k + \beta u_k + \gamma$. Running
OLS with $(x_{k+1}, x_k, u_k)$ triplets will give $\alpha$, $\beta$, and
$\gamma$. To obtain $K_s$, $K_v$, and $K_a$, solve the following system.
\begin{equation*}
  \begin{cases}
    \alpha = e^{-\frac{K_v}{K_a} T} \\
    \beta = \frac{1}{K_v}\left(1 - e^{-\frac{K_v}{K_a} T}\right) \\
    \gamma = \frac{K_s}{K_v}\left(e^{-\frac{K_v}{K_a} T} - 1\right)
  \end{cases}
\end{equation*}

Substitute $\alpha$ into the second and third equation.
\begin{equation}
  \begin{cases}
    \alpha = e^{-\frac{K_v}{K_a} T} \\
    \beta = \frac{1}{K_v}(1 - \alpha) \\
    \gamma = \frac{K_s}{K_v}(\alpha - 1)
  \end{cases} \label{eq:system}
\end{equation}

Solve the second equation in \eqref{eq:system} for $K_v$.
\begin{align}
  \beta &= \frac{1}{K_v}(1 - \alpha) \nonumber \\
  K_v \beta &= 1 - \alpha \nonumber \\
  K_v &= \frac{1 - \alpha}{\beta} \label{eq:kv}
\end{align}

Solve the first equation in \eqref{eq:system} for $K_a$.
\begin{align*}
  \alpha &= e^{-\frac{K_v}{K_a} T} \\
  \ln\alpha &= -\frac{K_v}{K_a} T \\
  K_a \ln\alpha &= -K_v T \\
  K_a &= -\frac{K_v T}{\ln\alpha}
\end{align*}

Substitute in $K_v$ from equation \eqref{eq:kv}.
\begin{align}
  K_a &= -\frac{\left(\frac{1 - \alpha}{\beta}\right) T}{\ln\alpha} \nonumber \\
  K_a &= -\frac{(1 - \alpha) T}{\beta \ln\alpha} \nonumber \\
  K_a &= \frac{(\alpha - 1) T}{\beta \ln\alpha}
\end{align}

Solve the third equation in \eqref{eq:system} for $K_s$.
\begin{align*}
  \gamma &= \frac{K_s}{K_v}(\alpha - 1) \\
  K_v \gamma &= K_s(\alpha - 1) \\
  K_s &= \frac{K_v \gamma}{\alpha - 1}
\end{align*}

Substitute in $K_v$ from equation \eqref{eq:kv}.
\begin{align}
  K_s &= \frac{\left(\frac{1 - \alpha}{\beta}\right) \gamma}{\alpha - 1}
    \nonumber \\
  K_s &= -\frac{\frac{1 - \alpha}{\beta} \gamma}{1 - \alpha}
    \nonumber \\
  K_s &= -\frac{\gamma}{\beta}
\end{align}

\section{Results}

The new characterization process is as follows.
\begin{itemize}
  \item Gather input voltage and velocity from a dynamic acceleration event
  \item Given an equation of the form
    $x_{k+1} = \alpha x_k + \beta u_k + \gamma$, perform OLS on
    $(x_{k+1}, x_k, u_k)$ triplets to obtain $\alpha$, $\beta$, and $\gamma$
\end{itemize}

The characterization gains are therefore
\begin{align}
  K_s &= -\frac{\gamma}{\beta} \\
  K_v &= \frac{1 - \alpha}{\beta} \\
  K_a &= \frac{(\alpha - 1) T}{\beta \ln\alpha} \\
      &\alpha > 0, \beta \neq 0 \nonumber
\end{align}

where $T$ is the sample period of the velocity data.

For $K_s$, $K_v$, and $K_a$ to be defined, the dataset must include robot
movement, and there must be a discernible relationship between a change in the
input voltage and a change in the velocity (in other words, the robot must have
accelerated).

\section{Bibliography}

\phantomsection
\section*{Online}
\addcontentsline{toc}{section}{Online}
\printbibliography[heading=bibempty,type=online]

\end{document}
